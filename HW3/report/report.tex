\documentclass{article}
\usepackage{amsmath}
\usepackage{hyperref}
\usepackage{graphicx}
\usepackage{adjustbox}
\newcommand{\tabincell}[2]{\begin{tabular}{@{}#1@{}}#2\end{tabular}}

\begin{document} %This is where document begins
\begin{titlepage}
\title{EE 232E \\Graphs and Network Flows\\Homework 3\\Winter 2016} 
\author{Liqiang Yu, Rongjing Bai, Yunwen Zhu\\
904592975, 204587519, 104593417}  %change your ID here
\date{05-01-2016}
\end{titlepage}

\maketitle
\newpage
\tableofcontents
\newpage

\section{Problem 3}
When transforming the undirected network into directed network, it's not trivial to choose the method. Here we have two options : (1) we can keep the number of edges unchanged and just remove the direction, however it will lead to the  non-simple network. (2) or we can merge the two directed edges between two nodes and make the new weight the geometric mean of the original weight. The fast greedy method can only be applied to the second method and label propagation method can be applied to both.
\subsection{Option 1}
When using label propagation method to calculate the community structure for option 1, we got 5 communities, the sizes of each community are shown in table


\end{document}
