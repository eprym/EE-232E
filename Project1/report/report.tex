\documentclass{article}
\usepackage{amsmath}
\usepackage{hyperref}
\usepackage{graphicx}
\usepackage{adjustbox}
\usepackage{caption}
\usepackage{subfigure}
\newcommand{\tabincell}[2]{\begin{tabular}{@{}#1@{}}#2\end{tabular}}
\begin{document} %This is where document begins
\begin{titlepage}
\title{EE 232E \\Graphs and Network Flows\\Project 1\\Winter 2016} 
\author{Liqiang Yu, Rongjing Bai, Yunwen Zhu\\
904592975, 204587519, 104593417}  %change your ID here
\date{05-14-2016}
\end{titlepage}
\maketitle
\newpage
\tableofcontents
\newpage

\section{Problem 4}
After deleting the core node 1, the personal network of core node 1 is shown in figure \ref{fig:p4_1}
\begin{figure}[htbp]
\centering
\includegraphics[width=.8\textwidth]{p4_1.png}
\caption{in degree distribution}
\label{fig:p4_1}
\end{figure}
After applying Fast Greedy, Edge Betweenness and Infomap algorithms to the personal network of core node 1 after the deletion, we can explore the community structures and compare them with the network before the deletion. The results are shown in figure \ref{fig:p4_2}. The left column are the community structures after applying three kinds of algorithms and the right column are the comparison results of the community distribution before and after the deletion.\\
\\
 From figure \ref{fig:p4_2} we can see that edge betweenness and infomap method almost produced the same results, but fast greedy produced the distribution that moved right a little bit. \\
 \\
 Moreover, the modularity results before and after the deletion are shown in table \ref{tb:p4}. From the results we can see that infomap produced exactly the same results, however fast greedy and edge betweenness both produced larger results after the deletion, which makes sense. Because the modularity is, quoted from wikipedia, "designed to measure the strength of division of a network into modules (also called groups, clusters or communities). Networks with high modularity have dense connections between the nodes within modules but sparse connections between nodes in different modules." After deleting the core node, which is pivot node of the its personal network, the network should be easier to divide into modules and therefore the modularity should increase.
\begin{figure}[htbp]
\centering
%\captionsetup{justification=centering,margin=2cm}

\subfigure{
\begin{minipage}[b]{0.4\textwidth}
\includegraphics[width=1\textwidth]{p4_2_fgc.png} \\
\caption*{The community structure for fast-greedy method}
\includegraphics[width=1\textwidth]{p4_2_ebc.png}\\
\caption*{The community structure for edge-betweenness method}
\includegraphics[width=1\textwidth]{p4_2_ifc.png}
\caption*{The community structure for infomap method}
\end{minipage}
}
\subfigure{
\begin{minipage}[b]{0.4\textwidth}
\includegraphics[width=1\textwidth]{density_compare_fgc.png} \\
\caption*{The comparison results for fast-greedy method}
\includegraphics[width=1\textwidth]{density_compare_ebc.png}\\
\caption*{The comparison results for edge-betweenness method}
\includegraphics[width=1\textwidth]{density_compare_ifc.png}
\caption*{The comparison results for infomap method}
\end{minipage}
}
\caption{}
\label{fig:p4_2}
\end{figure}

\begin {table}[htbp]
\caption{The modularity comparison results}
\begin{adjustbox}{center}
\label{tb:p4}
\begin{tabular}{|c|c|c|c|}
\hline
&fast-greedy &edge-betweenness &infomap\\
\hline
Before deletion&0.4131014&0.3533022&0.4180077\\
\hline
After deletion&0.4418533& 0.4161461& 0.4180077\\
\hline
\end{tabular}
\end{adjustbox}
\end{table}

\section{Problem 6}
In this problem, we choose to use the cluster coefficient and density to define two types of communities in the 41 personal networks. We try to find the community indices that have maximal and minimal results and compare them to draw the conclusion.
\subsection{Clustering Coefficient}
\subsubsection{Global Clustering Coefficient}
The global clustering coefficient is based on triplets of nodes. A triplet consists of three connected nodes. A triangle therefore includes three closed triplets. Therefore the global clustering coefficient is the number of closed triplets (or 3 x triplets) over the total number of triplets.
\begin{equation*}
C = \frac{3\;\times\;number\; of\; triangles}{number\; of\; connected\; triplets}
\end{equation*}
\subsubsection{Local Clustering Coefficient}
The local clustering coefficient of a vertex in a graph quantifies how close its neighbors are to being a clique. A graph $G=(V,E)$ formally consists of a set of vertices $V$ and a set of edges $E$ between them. A edge $e_{ij}$ connects vertex $v_i$ with $v_j$. The neighborhood $N_i$ for a vertex $v_i$ is defined as its immediately connected neighbors as follows
\begin{equation*}
N_i=\lbrace v_j:e_{ij}\in E \cup e_{ji} \in E \rbrace
\end{equation*} 
we define $k_i$ as the number of vertices in $N_i$. So the local clustering coefficient for undirected graph is
\begin{equation*}
C_i = \frac{2 \vert \lbrace e_{jk} : v_j, v_k \in N_i, e_{jk} \in E\rbrace\vert}{k_i(k_i-1)}
\end{equation*}
\subsubsection{Network average clustering coefficient}
As an alternative to the global clustering coefficient, the overall level of clustering in a network is measured as the average of the local clustering coefficients of all the vertices $n$
\begin{equation*}
\bar{C} = \frac{1}{n} \sum_{i=1}^nC_i
\end{equation*}
\subsection{Density} 
Network density describes the portion of the potential connections in a network that are actual connections. A potential network is a connection that could potentially exist between two nodes - regardless of whether or not it actually does. A network with high density has large possibility to be a clique.
\subsection{Resutls}
The results are shown in table \ref{tb:type}. We calculate both the cluster coefficient and density of each community that is larger than 10 in each personal network and store the community indices of the largest and smallest values.\\
\begin{table}[hbp]
\caption{The two type communities' indices}
\begin{adjustbox}{center}
\label{tb:type}
\begin{tabular}{|c|c|}
\hline
type1 cluster &6 3 2 4 1 1 1 2 1 2 1 1 2 2 2 2 1 3 2 1 1 1 2 3 1 2 1 2 3 2 3 1 1 2 1 1 2 1 2 1 2\\
\hline
type1 density &6 3 2 4 1 1 1 2 1 2 1 1 2 2 2 2 1 3 2 1 1 1 2 3 1 2 1 2 3 2 3 1 1 2 1 1 2 1 2 1 2\\
\hline
type2 cluster &7 4 3 5 4 3 2 1 3 1 2 3 1 1 1 1 2 1 3 2 8 2 1 1 2 1 2 1 1 1 1 2 2 1 2 2 1 2 1 2 1\\
\hline
type2 density &7 4 3 5 4 3 2 1 3 1 2 3 1 1 1 1 2 1 3 2 8 2 1 1 2 1 2 1 1 1 1 2 2 1 2 2 1 2 1 2 1\\
\hline
\end{tabular}
\end{adjustbox}
\end{table}
\\
Type 1 are indices with largest values and type 2 are indices with smallest values. From the table \ref{tb:type} we can see that clustering coefficient and density gave exactly the same results. So we can draw the conclusion that both clustering coefficient and density can distinguish two types of communities, one is densely connected inside, maybe a network of high school classmates and another sparselyl connected inside, maybe the fans of a sport team.
\end{document}
